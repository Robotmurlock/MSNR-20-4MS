\documentclass[bookmarks=true,bookmarksopen=true,pdfborder={0 0 0},pdfhighlight={/N},linkbordercolor={.5 .5 .5},implicit=false,unicode,xcolor={table}]{beamer}

\usepackage[T1]{fontenc}
\usepackage[utf8]{inputenc}

\usepackage[english,serbian]{babel}

%\usepackage[unicode]{hyperref}

\mode<presentation>
\setbeamercovered{transparent}
\setbeamertemplate{navigation symbols}{}

\usetheme{Frankfurt}
\usecolortheme{crane}

\usepackage{color}
%\usepackage[table]{xcolor}
\usepackage{url}

\usepackage{graphicx}
\usepackage{xspace}
\usepackage{fancyvrb}
\usepackage{listings}


\graphicspath{ {./Slike/} }
\usepackage{subcaption}
\captionsetup{compatibility=false}

\definecolor{mygreen}{rgb}{0,0.6,0}
\definecolor{mygray}{rgb}{0.5,0.5,0.5}
\definecolor{mymauve}{rgb}{0.58,0,0.82}

\lstset{ 
  backgroundcolor=\color{white},   % choose the background color; you must add \usepackage{color} or \usepackage{xcolor}; should come as last argument
  basicstyle=\scriptsize\ttfamily,        % the size of the fonts that are used for the code
  breakatwhitespace=false,         % sets if automatic breaks should only happen at whitespace
  breaklines=true,                 % sets automatic line breaking
  captionpos=b,                    % sets the caption-position to bottom
  commentstyle=\color{mygreen},    % comment style
  deletekeywords={...},            % if you want to delete keywords from the given language
  escapeinside={\%*}{*)},          % if you want to add LaTeX within your code
  extendedchars=true,              % lets you use non-ASCII characters; for 8-bits encodings only, does not work with UTF-8
  firstnumber=1,                % start line enumeration with line 1000
  frame=single,	                   % adds a frame around the code
  keepspaces=true,                 % keeps spaces in text, useful for keeping indentation of code (possibly needs columns=flexible)
  keywordstyle=\color{blue},       % keyword style
  language=C,                      % the language of the code
  morekeywords={*,...},            % if you want to add more keywords to the set
  numbers=left,                    % where to put the line-numbers; possible values are (none, left, right)
  numbersep=5pt,                   % how far the line-numbers are from the code
  numberstyle=\tiny\color{mygray}, % the style that is used for the line-numbers
  rulecolor=\color{black},         % if not set, the frame-color may be changed on line-breaks within not-black text (e.g. comments (green here))
  showspaces=false,                % show spaces everywhere adding particular underscores; it overrides 'showstringspaces'
  showstringspaces=false,          % underline spaces within strings only
  showtabs=false,                  % show tabs within strings adding particular underscores
  stepnumber=1,                    % the step between two line-numbers. If it's 1, each line will be numbered
  stringstyle=\color{mymauve},       % string literal style
  tabsize=2,	                   % sets default tabsize to 2 spaces
  title=\lstname                   % show the filename of files included with \lstinputlisting; also try caption instead of title
}

\setbeamertemplate{itemize item}[circle]
\setbeamertemplate{itemize subitem}[square]


\begin{document}
\title{Debagovanje sa LLDB-om}
\author{Momir Adžemović, Miloš Miković, Marko Spasić,\\ Mladen Dobrašinović\\~\\\small momir.adzemovic@gmail.com, spaskeasm@gmail.com,\\ \small milos.mikovicpos@gmail.com, dobrasinovic.mladen@gmail.com}
\subtitle{Prezentacija seminarskog rada}
\institute{Matematički fakultet}
\date{}


\begin{frame}
  
  \titlepage{}
  
\end{frame}

\begin{frame}{Uvod}
  
  \begin{itemize}
  	\item Zašto koristiti debager?
  	\item LLDB debager
  	\item Platforme na kojima radi
  	\item LLDB u poređenju sa drugim debagerima
  \end{itemize}

\end{frame}

\section{}
\begin{frame}{Debagovanje i bagovi uopšteno}
  \begin{itemize}
  \item Šta je debagovanje?
  \item Koji su koraci pri debagovanju?
  \item Šta je bag?
  \end{itemize}
  ,,Debagovanje je duplo teže od kodiranja, ako napišete kod na najlukaviji (odnosno najkomplikovaniji) način, po definiciji niste dovoljno pametni da ga debagujete.'' (Brian W. Kernighan)
\end{frame}

\begin{frame}[fragile]{Debager}

  Šta je debager?\\~\\

  Zašto debager?\\~\\
  
    \begin{lstlisting}
const char *str = "hello";
size_t length = strlen(str);
size_t i;
for(i = length - 1; i >= 0; i--)
  putchar(str[i]);\end{lstlisting}
\end{frame}


\begin{frame}{Šta je LLDB?}
  \begin{itemize}
  \item LLDB je napravljen sa ciljem da se stvori visoko-performantni debager sledeće generacije u okviru projekta LLVM.
  \item LLDB iskorištava funkcionalnosti Clang kompilatora i LLVM-a.
    (disassembly, parser, JIT kompilator, podrška za jezike)
    \begin{itemize}
    \item Podržava C++, C, Objective-C
    \end{itemize}
  \item LLDB pruža C++ i Python API-e i pruža podršku za skripting.
  \end{itemize}
\end{frame}

\begin{frame}[fragile]{Osnove LLDB debagera komandne linije}
  Sintaksna struktura komandi:
  {\color{mymauve}{
    \verb|<noun> <verb> [-option [value]] [argument [argument...]]|
  }}\\~\\

  
  Nove komande se mogu definisati sa:\\
  {\color{mymauve}{\verb|command alias|}}\\~\\

  Za istraživanje komandi u alatu se mogu koristiti {\color{mymauve}\verb|help|} i {\color{mymauve}\verb|apropos|}.
\end{frame}

\newcounter{row}
\setcounter{row}{0}
\begin{frame}[fragile]{Upoznavanje sa LLDB-om}
  \begin{table}[h!]
    \begin{center}
      \caption{Neke od osnovnih komandi LLDB-a.}
      \small
      \begin{tabular}{|>{\stepcounter{row}\arabic{row}.}l| >{\color{mymauve}}l|}
        \hline
	&\verb|process launch -- <argumenti>|
	\\ \hline
	&\verb|thread step-in|
	\\ \hline
	&\verb|thread step-inst-over|
	\\ \hline
	&\verb|breakpoint set --file 1.c --line 42|
	\\ \hline
	&\verb|breakpoint list|
	\\ \hline
	&\verb|breakpoint disable 1|
	\\ \hline
        &\verb|watchpoint set variable global_var|
        \\ \hline
        &\verb|watchpoint modify -c '(global_var == 5)'|
        \\ \hline
      \end{tabular}
    \end{center}
  \end{table}
\end{frame}

\begin{frame}{Platforme koje podržava LLDB}
	\begin{figure}
		
		\begin{subfigure}{2cm}
			\includegraphics[width=2cm,height=2cm]{windows_logo}
		\end{subfigure}
		\begin{subfigure}{2cm}
			\includegraphics[width=2cm,height=2cm]{linux_logo}
		\end{subfigure}
		\begin{subfigure}{2cm}
			\includegraphics[width=2cm,height=2cm]{mac_os}
		\end{subfigure}
		\begin{subfigure}{2cm}
			\includegraphics[width=2cm,height=2cm]{free_bsd}
		\end{subfigure}
	\end{figure}
\end{frame}

\begin{frame}{Funkcionalnosti}
	Uglavnom svuda:
	\begin{itemize}
		\item Backtracking 
		\item Breakpoints
		\item C++11
		\item Core file debugging
		\item Remote debugging
		\item Disassembly
		\item Expression evaluation {Windows ima problema}
	\end{itemize}
\end{frame}

\begin{frame}{Razvojna okruženja}
\begin{figure}
	
	\begin{subfigure}{2cm}
		\includegraphics[width=2cm,height=2cm]{clion_logo}
		\subcaption{Deo okruženja}
	\end{subfigure}
	\begin{subfigure}{2cm}
		\includegraphics[width=2cm,height=2cm]{xcode_logo}
		\subcaption{Deo okruženja}
	\end{subfigure}
	\begin{subfigure}{2cm}
		\includegraphics[width=2cm,height=2cm]{visual_code}
		\subcaption{Plugin}
	\end{subfigure}
	\begin{subfigure}{2cm}
		\includegraphics[width=2cm,height=2cm]{eclipse_logo}
		\subcaption{Plugin}
	\end{subfigure}
\end{figure}
\end{frame}

\begin{frame}{Poređenje sa ostalim popularnim debagerima}

	Kako bi programer izabrao debager potrebno je da odgovori na sledeća pitanja poštujući redosled:
	\begin{itemize}
		\item Koje debagere podržava operativni sistem?
		\item Da li okruženje koje koristi ima specijalnu podršku za odgovarajući debager?
		\item Koji debager mu najviše odgovara?
	\end{itemize}

	Konkurencija:
	\begin{itemize}
		\item GDB
		\item Visual Studio Debugger
	\end{itemize}
	
\end{frame}

\begin{frame}{Poređenje sa GDB i VSD}
	
	LLDB i GDB
	\begin{itemize}
		\item Sličan skup komandi i opcija
		\item GDB nudi više mogućnosti za sada (primer: single thread breakpoint)
		\item LLDB je lakše formalno ubaciti u softver (licenca)
		\item LLDB je moderniji (moderne biblioteke)
		\item LLDB nema maskotu kao GDB:
		\begin{figure}
			\includegraphics[width=10mm,height=5mm]{Slike/gdb_mascot.png}
		\end{figure}
	\end{itemize}
	
	LLDB i Visual Studio Debugger
	\begin{itemize}
		\item VSD nudi grafički interfejs
		\item LLDB ima bolju podršku za operativne sisteme
	\end{itemize}

\end{frame}

\begin{frame}{Zaključak}
\begin{itemize}
	\item Odličan uz MacOS
	\item Nema dodatnih podešavanja uz Clion i Xcode5
	\item Na Windows-u i dalje možda bolji Visual Studio Debager
	\item Deo LLVM projekta
	\item Razvoj brzo napreduje
\end{itemize}
\end{frame}

\begin{frame}{Literatura}
  \nocite{*}
  \bibliographystyle{unsrt}
  \bibliography{prezentacija}
\end{frame}

\end{document}
