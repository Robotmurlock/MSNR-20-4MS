% !TEX encoding = UTF-8 Unicode
\documentclass[a4paper]{article}

\usepackage{color}
\usepackage{url}
\usepackage[T2A]{fontenc} % enable Cyrillic fonts
\usepackage[utf8]{inputenc} % make weird characters work
\usepackage{graphicx}

\usepackage[english,serbian]{babel}
%\usepackage[english,serbianc]{babel} %ukljuciti babel sa ovim opcijama, umesto gornjim, ukoliko se koristi cirilica

\usepackage[unicode]{hyperref}
\hypersetup{colorlinks,citecolor=green,filecolor=green,linkcolor=blue,urlcolor=blue}

\usepackage{listings}

%\newtheorem{primer}{Пример}[section] %ćirilični primer
\newtheorem{primer}{Primer}[section]

\definecolor{mygreen}{rgb}{0,0.6,0}
\definecolor{mygray}{rgb}{0.5,0.5,0.5}
\definecolor{mymauve}{rgb}{0.58,0,0.82}

\lstset{ 
  backgroundcolor=\color{white},   % choose the background color; you must add \usepackage{color} or \usepackage{xcolor}; should come as last argument
  basicstyle=\scriptsize\ttfamily,        % the size of the fonts that are used for the code
  breakatwhitespace=false,         % sets if automatic breaks should only happen at whitespace
  breaklines=true,                 % sets automatic line breaking
  captionpos=b,                    % sets the caption-position to bottom
  commentstyle=\color{mygreen},    % comment style
  deletekeywords={...},            % if you want to delete keywords from the given language
  escapeinside={\%*}{*)},          % if you want to add LaTeX within your code
  extendedchars=true,              % lets you use non-ASCII characters; for 8-bits encodings only, does not work with UTF-8
  firstnumber=1000,                % start line enumeration with line 1000
  frame=single,	                   % adds a frame around the code
  keepspaces=true,                 % keeps spaces in text, useful for keeping indentation of code (possibly needs columns=flexible)
  keywordstyle=\color{blue},       % keyword style
  language=Python,                 % the language of the code
  morekeywords={*,...},            % if you want to add more keywords to the set
  numbers=left,                    % where to put the line-numbers; possible values are (none, left, right)
  numbersep=5pt,                   % how far the line-numbers are from the code
  numberstyle=\tiny\color{mygray}, % the style that is used for the line-numbers
  rulecolor=\color{black},         % if not set, the frame-color may be changed on line-breaks within not-black text (e.g. comments (green here))
  showspaces=false,                % show spaces everywhere adding particular underscores; it overrides 'showstringspaces'
  showstringspaces=false,          % underline spaces within strings only
  showtabs=false,                  % show tabs within strings adding particular underscores
  stepnumber=2,                    % the step between two line-numbers. If it's 1, each line will be numbered
  stringstyle=\color{mymauve},     % string literal style
  tabsize=2,	                   % sets default tabsize to 2 spaces
  title=\lstname                   % show the filename of files included with \lstinputlisting; also try caption instead of title
}

\begin{document}

\title{Debagovanje u LLDB-u\\ \small{Seminarski rad u okviru kursa\\Metodologija stručnog i naučnog rada\\ Matematički fakultet}}

\author{Momir Adžemovic, Miloš Miković, Marko Spasić,\\ Mladen Dobrašinović\\ \\ momir.adzemovic@gmail.com, spaskeasm@gmail.com,\\ milos.mikovicpos@gmail.com, dobrasinovic.mladen@gmail.com}

\date{1.~april 2020.}

\maketitle

\abstract{
U ovom tekstu je ukratko prikazana osnovna forma seminarskog rada. Obratite pažnju da je pored ove .pdf datoteke, u prilogu i odgovarajuća .tex datoteka, kao i .bib datoteka korišćena za generisanje literature. Na prvoj strani seminarskog rada su naslov, apstrakt i sadržaj, i to sve mora da stane na prvu stranu! Kako bi Vaš seminarski zadovoljio standarde i očekivanja, koristite uputstva i materijale sa predavanja na temu pisanja seminarskih radova. Ovo je samo šablon koji se odnosi na fizički izgled seminarskog rada (šablon koji \emph{morate} da koristite!) kao i par tehničkih pomoćnih uputstava. Pročitajte tekst pažljivo jer on sadrži i važne informacije vezane za zahteve obima i karakteristika seminarskog rada.}

\tableofcontents

\newpage

\section{Uvod}
\label{sec:uvod}

Kada budete predavali seminarski rad, imenujete datoteke tako da sadrže redni broj teme, temu seminarskog rada, kao i prezimena članova grupe. Precizna uputstva na temu imenovnja će biti data na formi za predaju seminarskog rada. Predaja seminarskih radova biće isključivo preko veb forme, a NE slanjem mejla. Link na formu će biti dat u okviru obaveštenja na strani kursa. Vodite računa da prilikom predavanja seminarskog rada predate samo one fajlove koji su neophodni za ponovno generisanje pdf datoteke. To znači da pomoćne fajlove, kao što su .log, .out, .blg, .toc, .aux i slično, \textbf{ne treba predavati}.

\section{Osnovna uputstva}
Vaš seminarski rad mora da sadrži najmanje jednu \textbf{sliku}, najmanje jednu \textbf{tabelu} i najmanje \textbf{sedam referenci} u spisku literature. Najmanje jedna slika treba da bude originalna i da predstavlja neke podatke koje ste Vi osmislili da treba da prezentujete u svom radu. Isto važi i za najmanje jednu tabelu. 	Od referenci, neophodno je imati bar jednu \textbf{knjigu}, bar jedan \textbf{naučni članak} iz odgovarajućeg časopisa i bar jednu adekvatnu \textbf{veb adresu}. 

\textbf{Dužina seminarskog rada treba da bude od 10 do 12 strana.} Svako prekoračenje ili potkoračenje biće kažnjeno sa odgovarajućim brojem poena. Eventualno, nakon strane 12, može se javiti samo tekst poglavlja \textbf{Dodatak} koji sadrži nekakav dodatni k\^{o}d, ali je svakako potrebno da rad može da se pročita i razume i bez čitanja tog dodatka. 

Ко жели, може да пише рад ћирилицом. У том случају, неопходно је да су инсталирани одговарајући пакети: texlive-fonts-extra, texlive-latex-extra, texlive-lang-cyrillic, texlive-lang-other. 

Nemojte koristiti stari način pisanja slova, tj ovo:
\begin{verbatim}
\v{s} i \v{c} i \'c ...
\end{verbatim}
Koristite direknto naša slova:	
\begin{verbatim}
š i č i ć ... 
\end{verbatim}


% Mladenov deo
\section{Upoznavanje sa LLDB-om}
LLDB podržava standardne funkcije debagovanja preko komandne linije i može se
koristiti kao debager u interaktivnom razvojnom okruženju. Konkretno, sa
debagerom pokrenutim nad programom prevedenim sa debug opcijama omogućava
se \cite{lldb_to_gdb_map}:

\begin{itemize}
\item{Aktiviranje procesa programa sa određenim argumentima komandne linije
(eng. command line arguments).}

\item{Korišćenje breakpoint-a (određenog reda ili funkcije u izvornom kodu pri
  kojima debager zaustavlja izvršavanje programa kada se stigne do odgovarajućeg
  dela izvršnog koda).}
  
\item{Korišćenje watchpoint-a (određene promenljive, takve da debager zaustavlja
proces ili nit kada se njeno stanje promeni).}

\item{Korišćenje dodatnih uslova nad breakpoint-ovima i watchpoint-ovima.}
  \begin{itemize}
  \item{Nastavljanje ili pokretanje programa.}
  \end{itemize}

\item{Pokretanje procesa red po red (sa ,,ulaženjem'' u funkciju ili bez).}

\item{Istraživanje promenljivih ili memorije procesa.}

\item{Izvršavanje proizvoljnog izraza nad stanjem procesa (npr. menjanje neke
promenljive na steku).}

\item{Istraživanje steka okvira poziva.}

\item{Izvršavanje drugih naprednih i raznih funkcija.}
\end{itemize}

Ono što ističe LLDB je omogućavanje korišćenja eksternih skripti za debagovanje
preko javnog API-a za Python, izršavanje proizvoljnog Python koda unutar
debagera \cite{lldb_python} (preko ugnježdenog interpretatora (eng. embedded
interpreter) i omogućavanje REPL (Read-Evaluate-Print-Loop) funkcija za
programske jezike zajedno sa mogućnostima debagovanja \cite{swift_lldb_repl}.

\subsection{LLDB interfejs komandne linije}
LLDB interfejs komandne linije (eng. command line interface) se aktivira pozivom
\verb|lldb| u ljusci (eng. shell) sa programom koji želimo debagovati kao
argumentom. Program komandne linije \verb|lldb| se odlikuje strukturisanom sintaksom
osnovnih komandi koja je sledećeg oblika \cite{lldb_tutorial}:
\begin{primer}
  \begin{footnotesize}
\begin{verbatim}

<imenica> <glagol> [-opcije [vrednost-opcije]] [argument [argument...]]
\end{verbatim}
  \end{footnotesize}
\end{primer}
U ovakvom obliku, imenica se zove i komanda, a glagol potkomanda. Postoje i
skraćenice (eng. alias) za komande koje mogu odstupati od ovog oblika. Upravo
zato što je ovaj format komandi jako strukturisan mogu nam biti pogodni skraćeni
oblici komandi koji mogu biti sličniji onome što je poznato korisnicima drugih
debagera \cite{apple_lldb_comms}. U tabeli \ref{tab:tabela3} su date neke od
osnovnih komandi kao primer korišćenja interfejsa i reprezentativni prikaz
širokog skupa mogućnosti LLDB-a koji nije naveden u potpunosti u ovom
radu. Posebno se ističu komande \verb|help| i \verb|apropos|, koje mogu biti
korisne početnicima u korišćenju ovog alata.

\begin{table}[h!]
  \begin{center}
    \caption{Upotreba interfejsa komandne linije LLDB-a \cite{lldb_to_gdb_map}\cite{lldb_tutorial}.}
    \small
    \begin{tabular}{|r|p{5cm}|}
      \hline
      \verb|process launch -- <argumenti>|
      & Pokreće izabrani program sa datim argumentima. \\ \hline
      \verb|thread step-in|
      & U trenutnoj niti nastavlja izvršavanje programa sledeće instrukcije izvornog koda, ulazeći u pozive funkcija.  \\ \hline
      \verb|thread step-inst-over|
      & U trenutnoj niti nastavlja izvršavanje programa sledeće instrukcije izvršnog koda, ne ulazeći u pozive funkcija. \\ \hline
      \verb|breakpoint set --file 1.c --line 42|
      & Postavlja breakpoint na red 42 u izvornom kodu programa 1.c. \\ \hline
      \verb|breakpoint list|
      & Ispisuje postojeće breakpoint-ove debagera. \\ \hline
      \verb|breakpoint disable 1|
      & Deaktivira breakpoint 1. \\ \hline
      \verb|apropos <ključna_reč>|
      & Traži u pomoći za upotrebu komandi (eng. command help) datu ključnu reč. \\ \hline
      \verb|help|
      & Štampa pomoć za komande. (help se može koristiti i za nalaženje pomoći za upotrebu potkomandi određene komande \cite{apple_lldb_comms}). \\ \hline
    \end{tabular}
    \label{tab:tabela3}
  \end{center}
\end{table}


\section{Engleski termini i citiranje}	
\label{sec:termini_i_citiranje}

Na svakom mestu u tekstu naglasiti odakle tačno potiču informacije. Uz sve novouvedene termine u zagradi naglasiti od koje engleske reči termin potiče. 

Naredni primeri ilustruju način uvođenja enlegskih termina kao i citiranje.

\begin{primer}
Problem zaustavljanja (eng.~{\em halting problem}) je neodlučiv.
\end{primer}

\begin{primer}
Za prevođenje programa napisanih u programskom jeziku C može se koristiti GCC kompajler.
\end{primer}

\begin{primer}
 Da bi se ispitivala ispravost softvera, najpre je potrebno precizno definisati njegovo ponašanje. 
\end{primer}

Reference koje se koriste u ovom tekstu zadate su u datoteci {\em seminarski.bib}. Prevođenje u pdf format u Linux okruženju može se uraditi na sledeći način:
\begin{verbatim}
pdflatex TemaImePrezime.tex 
bibtex TemaImePrezime.aux 
pdflatex TemaImePrezime.tex 
pdflatex TemaImePrezime.tex 
\end{verbatim}
Prvo latexovanje je neophodno da bi se generisao {\em .aux} fajl. {\em bibtex} proizvodi odgovarajući {\em .bbl} fajl koji se koristi za generisanje literature. 
Potrebna su dva prolaza (dva puta pdflatex) da bi se reference ubacile u tekst (tj da ne bi ostali znakovi pitanja umesto referenci). Dodavanjem novih referenci potrebno je ponoviti ceo postupak.  











Broj naslova i podnaslova je proizvoljan. Neophodni su samo Uvod i Zaključak. Na poglavlja unutar teksta referisati se po potrebi. 
\begin{primer}
U odeljku \ref{sec:naslov1} precizirani su osnovni pojmovi, dok su zaključci dati u odeljku \ref{sec:zakljucak}.
\end{primer}

Još jednom da napomenem da nema razloga da pišete:
\begin{verbatim}
\v{s} i \v{c} i \'c ...
\end{verbatim}
Možete koristiti srpska slova
\begin{verbatim}
š i č i ć ... 
\end{verbatim}



\section{Slike i tabele}
\label{slike_i_tabele}

Slike i tabele treba da budu u svom okruženju, sa odgovarajućim naslovima, obeležene labelom da koje omogućava referenciranje. 

\begin{primer} Ovako se ubacuje slika. Obratiti pažnju da je dodato i 
\begin{verbatim}
\usepackage{graphicx}
\end{verbatim}

Na svaku sliku neophodno je referisati se negde u tekstu. Na primer, na slici \ref{fig:pande} prikazane su pande. 
\end{primer}

\begin{primer} I tabele treba da budu u svom okruženju, i na njih je neophodno referisati se u tekstu. Na primer, u tabeli \ref{tab:tabela1} su prikazana različita poravnanja u tabelama.

\begin{table}[h!]
\begin{center}
\caption{Razlčita poravnanja u okviru iste tabele ne treba koristiti jer su nepregledna.}
\begin{tabular}{|c|l|r|} \hline
centralno poravnanje& levo poravnanje& desno poravnanje\\ \hline
a &b&c\\ \hline
d &e&f\\ \hline
\end{tabular}
\label{tab:tabela1}
\end{center}
\end{table}

\end{primer}

\section{K\^{o}d i paket listings}
Za ubacivanje koda koristite paket \textbf{listings}:
\url{https://en.wikibooks.org/wiki/LaTeX/Source_Code_Listings}

\begin{primer}
Primer ubacivanja koda za programski jezik Python dat je kroz listing \ref{simple}. Za neki drugi programski jezik, treba podesiti odgvarajući programski jezik u okviru defnisanja stila.
\end{primer}
\begin{lstlisting}[caption={Primer ubacivanja koda u tekst},frame=single, label=simple]
# This program adds up integers in the command line
import sys
try:
    total = sum(int(arg) for arg in sys.argv[1:])
    print 'sum =', total
except ValueError:
    print 'Please supply integer arguments'
\end{lstlisting}


\section{Prvi naslov}
\label{sec:naslov1}


Ovde pišem tekst. 
Ovde pišem tekst. 
Ovde pišem tekst. 
Ovde pišem tekst. 
Ovde pišem tekst. 
Ovde pišem tekst. 
Ovde pišem tekst. 
Ovde pišem tekst. 


\subsection{Prvi podnaslov}
\label{subsec:podnaslov1}

Ovde pišem tekst. 
Ovde pišem tekst. 
Ovde pišem tekst. 
Ovde pišem tekst. 
Ovde pišem tekst. 
Ovde pišem tekst. 
Ovde pišem tekst. 

\subsection{Drugi podnaslov}
\label{subsec:podnaslov2}

Ovde pišem tekst. 
Ovde pišem tekst. 
Ovde pišem tekst. 
Ovde pišem tekst. 
Ovde pišem tekst. 
Ovde pišem tekst. 


\subsection{... podnaslov}
\label{subsec:podnaslovN}

Ovde pišem tekst. 
Ovde pišem tekst. 
Ovde pišem tekst. 
Ovde pišem tekst. 
Ovde pišem tekst. 
Ovde pišem tekst. 

\section{Poredjenje sa drugim popularnim debagerima}
\label{sec:naslovN}

Potrebno je naglasiti da pri poređenju različitih debagera ne možemo objektivno odrediti koji je debager najbolji, jer to dosta zavisi koji se operativni sistem koristi, a i samih preferenci korisnika. Visual Studio Code, jedan od popularnijih editora, koristi LLDB, GDB i VSD za C++ u zavisnosti od od operativnog sistema na kojem je instaliran \cite{vsc_support}:
\begin{itemize}
	\item \textbf{Linux}: GDB
	\item \textbf{macOS}: LLDB or GDB
	\item \textbf{Windows}: the Visual Studio Windows Debugger or GDB (using Cygwin or MinGW)
\end{itemize}

\begin{figure}[h!]
	\begin{center}
		\includegraphics[width=100mm,height=50mm]{Slike/tabela_poredjenje1.png}
	\end{center}
	\caption{LLDB, GDB, Visual Studio Debugger \cite{gdb}\cite{lldb}\cite{vsd}}
	\label{fig:tabela_poredjenje1}
\end{figure}

\subsection{Poređenje: GDB i LLDB}
\label{subsec: GDB i LLDB}

Debuger GDB predstavlja standard za GNU sisteme (ne striktno samo za GNU) \cite{gdb}. Ako se proverava kvalitet debagera LLDB, onda u potpunosti ima smisla upoređivati ga prvo sa GDB debagerom kao jednim od najpopularnijih debagera. Debager LLDB u debagovanju velikih programa pokazuje bolje performanse od GDB debagera i ima dobar korisnički interfejs \cite{lldb_project_blog}. Način korišćenje ova dva debagera je veoma sličan i skup komandi se većinom poklapa. Postoji zvaničan rečnik koji prevodi komande iz GDB u LLDB \cite {lldb_to_gdb_map}. Novije verzije GDB podržavaju MacOS, ali u proteklih par godina se pretežno koristio LLDB kao glavni debager za MacOS.

\subsection{Visual Studio Debugger i LLDB}
\label{subsec: Visual Studio Debugger i LLDB}

VIsual studio debugger je takođe jedan od poznatijih debagera koji možemo da upoređujemo sa LLDB-om. Prednost VSD u odnosu na LLDB je u tome što VSD nudi grafički point-and-click korisnički interfejs, a prednost LLDB je u broju operativnih sistema koji za koju ima podršku \cite{vsd}.

\section{Zaključak}
\label{sec:zakljucak}

Ovde pišem zaključak. 
Ovde pišem zaključak. 
Ovde pišem zaključak. 
Ovde pišem zaključak. 
Ovde pišem zaključak. 
Ovde pišem zaključak. 
Ovde pišem zaključak. 
Ovde pišem zaključak. 
Ovde pišem zaključak. 
Ovde pišem zaključak. 
Ovde pišem zaključak. 
Ovde pišem zaključak. 


\addcontentsline{toc}{section}{Literatura}
\appendix
\bibliography{seminarski} 
\bibliographystyle{plain}

\appendix
\section{Dodatak}
Ovde pišem dodatne stvari, ukoliko za time ima potrebe.
Ovde pišem dodatne stvari, ukoliko za time ima potrebe.
Ovde pišem dodatne stvari, ukoliko za time ima potrebe.
Ovde pišem dodatne stvari, ukoliko za time ima potrebe.
Ovde pišem dodatne stvari, ukoliko za time ima potrebe.


\end{document}
